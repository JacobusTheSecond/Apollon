\documentclass[11pt, a4paper,draft]{report}
\usepackage[ngerman]{babel} 
\usepackage[T1]{fontenc}
\usepackage{lmodern}
\usepackage[utf8]{inputenc}
\usepackage{tabularx}
\usepackage{mathtools}
\usepackage{color}
%%%%%%%%%%%%%%%%%%%%%%%%%%%%%%%%%%%%%%%%%%%%%%%%
% load packages
\usepackage{amsmath}   % for more basic mathematical symbols
\usepackage{amssymb}   % for more mathematical symbols
\usepackage{amsthm}
\usepackage{tikz-cd}
\usepackage{tikz}
\usetikzlibrary{3d,calc,intersections}
\usetikzlibrary{arrows.meta} 
\usetikzlibrary{decorations.markings}
\usetikzlibrary{patterns}
\usepackage{dsfont}
\usepackage{ stmaryrd }
\usepackage{caption}
\usepackage[final]{listings}  


\newtheorem{theorem}{Satz}
\newtheorem{lemma}{Lemma}
\newtheorem*{remark}{Anmerkung}
\newtheorem{definition}{Definition}
\newcommand{\bK}{\mathbb{K}}
\newcommand{\bN}{\mathbb{N}}
\newcommand{\bZ}{\mathbb{Z}}
\newcommand{\bR}{\mathbb{R}}
\newcommand{\bC}{\mathbb{C}}
\newcommand{\bS}{\mathbb{S}}
\newcommand{\bA}{\mathbb{A}}
\newcommand{\bB}{\mathbb{B}}
\newcommand{\bD}{\mathbb{D}}
\newcommand{\bE}{\mathbb{E}}
\newcommand{\bT}{\mathbb{T}}
\newcommand{\bQ}{\mathbb{Q}}

\newcommand{\im}{\textrm{im}}
%%%%%%%%%%%%%%%%%%%%%%%%%%%%%%%%%%%%%%%%%%%%%%%%%%%%%%%

\renewcommand{\figurename}{Abb.}

\newcolumntype{L}[1]{>{\raggedright\arraybackslash}p{#1}} 
%Linksbündig mit vorgegebener Breite
\newcolumntype{C}[1]{>{\centering\arraybackslash}p{#1}} 
%Zentriert mit vorgegebener Breite
\newcolumntype{R}[1]{>{\raggedleft\arraybackslash}p{#1}} 
%Rechtsbündig mit vorgegebener Breite

\usepackage{BA_Titelseite}

%Namen des Verfassers der Arbeit
\author{Jacobus Leander Conradi\\Vincent Rheinthal}
%Geburtsdatum des Verfassers
\geburtsdatum{27. Februar 1998}
%Gebortsort des Verfassers
\geburtsort{Grevenbroich}
%Datum der Abgabe der Arbeit
\date{\today}

%Name des Betreuers
% z.B.: Prof. Dr. Peter Koepke
\betreuer{Betreuer: Prof. Dr. Carl-Friedrich Bödigheimer}
%Name des Instituts an dem der Betreuer der Arbeit tätig ist.
\zweitgutachter{Zweitgutachter: Dr. Lennart Meier}
%z.B.: Mathematisches Institut
\institut{Mathematisches Institut}
%Titel der Bachelorarbeit
\title{4-Mannigfaltigkeiten mit vorgegebenen Fundamentalgruppen}
%Do not change!
\ausarbeitungstyp{Bachelorarbeit Mathematik}

\usepackage{afterpage}

\newcommand\blankpage{%
	\null
	\thispagestyle{empty}%
	\addtocounter{page}{-1}%
	\newpage}

\definecolor{dkgreen}{rgb}{0,0.6,0}
\definecolor{gray}{rgb}{0.5,0.5,0.5}
\definecolor{mauve}{rgb}{0.58,0,0.82}

\lstset{language=Java,
	aboveskip=2mm,
	belowskip=2mm,
	showstringspaces=false,
	columns=flexible,
	basicstyle={\small\ttfamily},
	numbers=none,
	numberstyle=\tiny\color{gray},
	keywordstyle=\color{blue},
	commentstyle=\color{dkgreen},
	stringstyle=\color{mauve},
	breaklines=true,
	breakatwhitespace=true,
	tabsize=3
}


\begin{document} 
	\maketitle
	
	\thispagestyle{plain}
	\pagenumbering{Roman}
	\section*{Vorwort}\addcontentsline{toc}{chapter}{Vorwort}
	
	\newpage
	\clearpage
	\thispagestyle{empty}\mbox{}
	\newpage
	\addtocounter{page}{-1}
	\tableofcontents 
	\addcontentsline{toc}{chapter}{Inhaltsverzeichnis}
	\vfil\null
	\clearpage
	\thispagestyle{empty}\mbox{}
	\clearpage
	\pagenumbering{arabic}
	
	\chapter{Problemstellung und Definition}
	\section{Einführung}
	In dieser Arbeit wollen wir uns dem Vergleich verschiedener Bilddaten widmen. Hierfür betrachten wir das Konzept der persistenten Homologie. Persistente Homologie lässt sich am besten beschreiben, als das ''Erkennen von Formen mit zusammengekniffenen Augen''. Wir wollen also einen komplex aussehenden Datensatz mit diesem Ansatz in grundlegende geometrische Formen zurückführen. Dieser Ansatz ist insbesondere auch sehr intuitiv und nah an dem unterbewussten Erkennen von Strukturen, die einen Menschen auszeichnet.\\
	Wir wollen also zuerst ein Persistenzdiagramm zu einer gegebenen Punktmenge $X$ berechnen und danach ein geeignetes Abstandsmaß für solche Diagramme finden.
	
	\section{Persistente Homologie}
	
	Um Homologien zu berechnen benötigen wir zunächst einmal eine Mannigfaltigkeit, oder im Fall eines Computers bzw. einer notwendigerweise diskreten Darstellung, einen Zell-Komplex. Sei also nun im allgemeinen eine Punktmenge $X\subset \bR^d$ gegeben. Zuerst wollen wir den Čech-Komplex definieren.
	
	\begin{definition}[Čech-Komplex]
		Sei $X\subset \bR^d$ eine endliche Menge und $\varepsilon>0$ gegeben. Dann konstruieren wir $Č_\varepsilon(X)$ wie folgt. Das $0$-Skellet von $Č_\varepsilon(X)$ ist $X$ und für jede Menge $\sigma\subset X$ ist $\sigma$ im $|\sigma|$-Skellet von $Č_\varepsilon(X)$, falls der Schnitt der Bälle mit Radius $\varepsilon$ um alle Punkte in $\sigma$: $\bigcap_{x\in\sigma}B_\varepsilon(x)$ nicht leer ist.
	\end{definition}

	Man bemerke zunächst, dass für jedes endliche $X$ ein $\mathcal{E}>0$ existiert, sodass für alle $\varepsilon\geq\mathcal{E}$ gilt, dass $Č_\varepsilon(X) = Č_\mathcal{E}(X)$. Nenne diesen Zellkomplex, an dem der Čech-Komplex stabilisiert $C(X):=Č_\mathcal{E}(X)$. Wir definieren nun eine Filtration $f:C(X)\rightarrow\bR$,$\sigma\mapsto\min\{\varepsilon|\sigma\in Č_\varepsilon(X)\}$.\\
	Weiterhin bemerke man, dass das Bild von $f$ endlich ist. Sei $\im(f)=\{\varepsilon_i|0\leq i\leq N\}$. Dann erhalten wir eine Sequenz von Zellkomplexen:
	$$Č_{\varepsilon_1}(X)\hookrightarrow Č_{\varepsilon_2}(X)\hookrightarrow \ldots \hookrightarrow Č_{\varepsilon_{N-1}}(X)\hookrightarrow Č_{\varepsilon_N}(X).$$
	Mit einer solchen Kette können wir nun persistente Homologie definieren.
	\begin{definition}[persistente Homologie]
		Gegeben eine Sequenz von Zellkomplex Einbettungen: $$X_1\xhookrightarrow{\iota_1} X_2\xhookrightarrow{\iota_2} \ldots\xhookrightarrow{\iota_{n-2}} X_{n-1}\xhookrightarrow{\iota_{n-1}} X_n$$
		Betrachten wir die implizierte Homologie Sequenz für jedes $k$:
		$$H_k(X_1)\xrightarrow{\iota_{1}^*} H_k(X_2)\xrightarrow{\iota_{2}^*}\ldots \xrightarrow{\iota_{n-2}^*} H_k(X_{n-1})\xrightarrow{\iota_{n-1}^*} H_k(X_n).$$
		Für $0\leq i<j\leq n$ sei $\mathcal{L}_i^j(X)$ die Menge der Erzeuger $\sigma$, die im Cokernel von $\iota_{i-1}^*$ liegen, sodass das Bild von $\sigma$ in $H_{j-1}(X)$ nicht null ist, aber es im Kern von $\iota_{j-1}$ liegt. Diese Familie $\{\mathcal{L}_i^j(X)|0\leq i<j\leq n\}$ nennen wir die peristente Homologie von X.
	\end{definition}

	\begin{figure}
		\centering
	\begin{tikzpicture}
	\filldraw (0,0) coordinate(a) circle (3pt);
	\filldraw (-1,1) coordinate(b) circle (3pt);
	\filldraw (0.3,0.9) coordinate(c) circle (3pt);
	\filldraw (0,-1.3) coordinate(d) circle (3pt);
	\filldraw (-1,-2) coordinate(e) circle (3pt);
	\filldraw (-2,-1.5) coordinate(f) circle (3pt);
	\filldraw (-3,-0.5) coordinate(g) circle (3pt);
	\filldraw (-2,0.5) coordinate(h) circle (3pt);
	
	\filldraw[opacity=0.3] (a) circle (0.8);
	\filldraw[opacity=0.3] (b) circle (0.8);
	\filldraw[opacity=0.3] (c) circle (0.8);
	\filldraw[opacity=0.3] (d) circle (0.8);
	\filldraw[opacity=0.3] (e) circle (0.8);
	\filldraw[opacity=0.3] (f) circle (0.8);
	\filldraw[opacity=0.3] (g) circle (0.8);
	\filldraw[opacity=0.3] (h) circle (0.8);
	\end{tikzpicture}
	\begin{tikzpicture}
	\filldraw (0,0) coordinate(a) circle (3pt);
	\filldraw (-1,1) coordinate(b) circle (3pt);
	\filldraw (0.3,0.9) coordinate(c) circle (3pt);
	\filldraw (0,-1.3) coordinate(d) circle (3pt);
	\filldraw (-1,-2) coordinate(e) circle (3pt);
	\filldraw (-2,-1.5) coordinate(f) circle (3pt);
	\filldraw (-3,-0.5) coordinate(g) circle (3pt);
	\filldraw (-2,0.5) coordinate(h) circle (3pt);
	
	\draw (a) -- (b);
	\draw (a) -- (c);
	\draw (c) -- (b);
	
	\filldraw [opacity=0.3] (a) -- (b) -- (c);
	
	\draw (a) -- (d);
	\draw (d) -- (e);
	\draw (e) -- (f);
	\draw (f) -- (g);
	\draw (g) -- (h);
	\draw (h) -- (b);
	\end{tikzpicture}
	\caption{\textit{Ein Čech-Komplex für ein gegebenes $\varepsilon$ mit dem erzeugenden Datensatz.}}
	\label{fig1}
	\end{figure}

	Die persistente Homologie speichert also das Leben und Sterben verschiedener Erzeuger während dem Erhöhen der Grenze für Elemente in der Zellkomplexfiltration.\\
	Da wir uns in dieser Arbeit auf $2$-dimensionale Punktmengen $X$ beschränken, genügt es, den Zellkomplex $C(X)$ auf sein $2$-Komplex zu beschränken, da alle höheren Homologien verschwinden.
	
	Man kann das Berechnen dieser Homologien für $2$-dimensionale Daten wir folgt geometrisch definieren. Man lässt um jeden Punkt aus $X$ gleichmäßig Bälle wachsen, vereinigt diese und berechnet die Homologie. Hierbei entstehen zwischendurch Kreise, also Erzeuger der ersten Homologie, die bei großem $\varepsilon$ wieder verschwinden.
	\subsection{Persistenz Diagramme}
	
	Um jetzt mit persistenter Homologie arbeiten zu können wollen wir diese Visualisieren. Dafür kann man sich zum Beispiel Persistenz Diagramme angucken. Hierbei wollen wir für jeden Erzeuger der zu einem bestimmten Zeitpunkt $t_1$ geboren wird und zu einem Zeitpunkt $t_2$ stirbt, den Punkt $(t_1,t_2)$ in $\bR^2$ markieren.\\
	\begin{figure}
		\centering
		\fbox{
			\begin{tikzpicture}
		% Achsen zeichnen
		\draw[->,thick] (0,0) -- (5,0) node[right] {$x$};
		\draw[color=blue] (0,0) -- (5,5);
		\draw[->,thick] (0,0) -- (0,5) node[above] {$y$};
		%Punkte einzeichnen:
		
		\filldraw[color=red] (0,0.3) coordinate(a) circle (3pt);
		\filldraw[color=red] (0,1) coordinate(a) circle (3pt);
		\filldraw[color=red] (0,1.3) coordinate(a) circle (3pt);
		\filldraw[color=red] (0,3) coordinate(a) circle (3pt);
		\filldraw[color=red] (0,4) coordinate(a) circle (3pt);
		\filldraw[color=red] (0,3.3) coordinate(a) circle (3pt);
		\filldraw[color=red] (0,2) coordinate(a) circle (3pt);
		\filldraw[color=red] (0,2.2) coordinate(a) circle (3pt);
		\filldraw[color=red] (0,2.6) coordinate(a) circle (3pt);
		
		
		\filldraw[color=green] (2,3) coordinate(a) circle (3pt);
		\filldraw[color=green] (1,1.1) coordinate(a) circle (3pt);
		\filldraw[color=green] (3,4) coordinate(a) circle (3pt);
		
		\end{tikzpicture}
	}
	\caption{\textit{Ein Persistenzdiagramm. Rote Punkte gehören zur nullten, und grüne Punkt zur ersten Homologie}}
\label{fig2}
	\end{figure}

	 Man bemerke, dass die so markierten Punkte immer oberhalb der Diagonale $\Delta=\{(x,x)\in\bR^2|x\in\bR\}$ liegen, wobei Punkte die direkt an der Diagonalen liegen, Erzeuger repräsentieren, die schnell nach der Geburt sterben, und Punkte die weit von der Diagonalen entfernt sind, Erzeuger repräsentieren, die sehr lange leben. In Abb. \ref{fig2} sieht man ein Persistenzdiagramm, das zu drei unterschiedlichen Zeiten Kreise - also Erzeuger der ersten Homologie - aufweist. Des weiteren kann man sehen, dass die Erzeuger der nullten Homologie - also einer pro Zusammenhangskomponente zu jedem gegenen Zeitpunkt - alle zum Zeitpunkt 0 geboren werden. 
	\section{Wasserstein Distanz}
	
	Seien also nun zwei solche Persistenzdiagramme gegeben. Um ein Abstandsmaß zu definieren, betrachten wir die durch die Wahrscheinlichkeitstheorie definierte Wassersteindistanz. Hierbei handelt es sich um den ''minimalen Aufwand, eine Wahrscheinlichkeitsverteilung zu einer anderen zu bewegen''. In unserer konkreten Anwendung definieren wir diese wie folgt.
	
	\begin{definition}[Wassersteindistanz]
		Seien $X$ und $Y$ zwei Punktmengen gegeben, mit $|X|=|Y|$. Dann definieren wir die Wassersteindistanz
		$$W_p(X,Y) = \min_{\varphi:X\rightarrow Y}\bigg(\sum_{x\in X}||x-\varphi(x)||^p\bigg)^{\frac{1}{p}},$$
		wobei über alle Bijektionen $\varphi:X\rightarrow Y$ minimiert wird und $||\cdot||$ die $\ell_2$-Norm in $\bR^2$ ist.
	\end{definition}

	Ein Spezialfall $W_\infty(\cdot,\cdot)$ nennen wir die Bottleneckdistanz.\\
	Man bemerke die starke Einschränkung, dass $|X| = |Y|$ sein muss. Diese wollen wir geeignet anpassen, sodass auch unterschiedliche große Mengen miteinander verglichen werden können und unterschiedliche Herangehensweisen qualitativ vergleichen.
	
	\section{Feature Proposals}
	
	Als letzten Schritt wollen wir Methoden untersuchen, um zu einem gegebenen Bild eine representierende Punktemenge $X$ zu finden, um diese als Eingabe für das Berechnen der persistenten Homologie zu benutzten. Auch hier wollen wir unterschiedliche Methoden vergleichen und im Zusammenhang mit der persistenten Homologie vergleichen.
	
	\chapter{Persistente Homologie}
	
	In diesem Kapitel wollen wir die Implementierung zur Berechnung des Čech-Komplexes, sowie die Berechnung der persistenten Homologie widmen.
	
	\section{Čech-Komplex}
	
	Zum Berechnen des Čech-Komplexes betrachten wir zunächst einmal das Voronoi Diagramm für gegebene Punkte $X$ im $2$-dimensionalen Raum. Hierbei wird $\bR^2$ in Regionen unterteilt, sodass in jeder Region genau ein Punkt $x$ aus $X$ liegt, und für jeden anderen Punkt $y$ aus dieser Region ist $x$ der näheste Punkt aus $X$ an $y$.\\
	Aus dem Voronoi Diagramm können wir den Čech-Komplex extrahieren. Hierbei fügen wir zu dem Zellkomplex eine Kante zwischen $x$ und $y$ aus $X$ hinzu, wenn die Voronoi-Regionen von $x$ und $y$ sich berühren. Die Distanz von $x$ und $y$ gibt uns außerdem den Filtrationswert $f(\{x,y\}) = \frac{||x-y||}{2}$ der Kante. Fügt man alle diese Kanten hinzu, erhält man den planaren Dual-Graph, die DeLauney-Triangulierung der Punkte aus $X$.\\
	Komplizierter wird es bei Flächen, also $2$-Zellen. Hierbei müssen wir eine wichtige Fallunterscheidung treffen. Falls sich mehr als $2$ Voronoi-Regionen in einem Punkt berühren, korrespondiert dies zu einem Kreis und des weiteren einer $2$-Zelle im Čech-Komplex. Doch was genau ist der Filtrationswert. Hier kommt die folgende Fallunterscheidung ins Spiel. Betrachtet man die konvexe Hülle der Punkte $x_1,\ldots,x_k\in X$, deren Voronoi Regionen sich berühren, kann es passieren, dass der Punkt, indem sich die Voronoi-Regionen sich berühren, innerhalb oder außerhalb des Polytops liegen. Falls der Punkt $v$ innerhalb des Polytops liegt, muss der Filtrationswert als der Abstand von $v$ zu allen $x_i$, die alle gleich sind, gewählt werden. Denn sobald dieser Wert überschritten wird, ''füllt'' sich das Loch in der geometrischen Präsentation mit den Bällen, also muss die Fläche eingefügt werden.\\
 	\lstset{language=Java}
 	\begin{figure}
 		\begin{lstlisting}[frame=single]
 		public class Voronoi {
	 		//stores voronoi vertices, where more than 2 regions touch
	 		private PointD[] vertices = null;
	 		private VEdge[] edges = null;
	 		...
	 		private void compute(int width, int height) {
		 		VoronoiResults results = org.kynosarges.tektosyne.geometry.Voronoi.findAll(
		 		sites, new RectD(0, 0, width, height));
		 		vertices = results.voronoiVertices;
		 		//transform output of library to our own data types
		 		...
	 		}
	 		...
 		}
 		public class ActionGenerator {
	 		@NotNull
	 		public List<Action> generate() {
		 		//generates the list of elements added to the cell complex sorted by their filtration values
		 		...
		 		voronoi.forEachVertex(this::computeVertex);
		 		voronoi.forEachEdge(this::computeEdge);
		 		actions.sort(Action::compareTo);
		 		return actions;
		 	}
		 	private void computeVertex(@NotNull PointD vertex, int index) {
		 		//create list of actions given the Voronoi Diagram
		 		...
		 		VEdge[] edges = voronoi.getEdges(edgeIndices);
		 		PointD[] sites = getSites(edges);
		 		if (Util.isInside(vertex, sites)) {
		 			actions.add(new FaceAction(...));
		 			return;
		 		}
		 		...
		 		actions.add(new EdgeFaceAction(...));
	 		}
 		}
 		\end{lstlisting}
 		\caption{\textit{Codesnippet of the generation ofzur Generierung von $C(X)$}}
 		\label{fig3}
 	\end{figure}
 	Im anderen Fall gibt es keinen Zeitpunkt, wo der durch die Punkte $x_1,\ldots,x_k$ gegebene Kreis
 	im in der geometrischen Repräsentierung ''um ein Loch'' ist, sodass die Fläche zu dem Zeitpunkt eingefügt werden muss, wo die letzte Kante den Kreis in $x_1,\ldots,x_k$ schließt. Mit anderen Worten, der Filtrationswert ist $max_{\{x_i,x_y\}\in C(X)}||x_i-x_j||$. Diese $2$-Zellen wollen wir \textit{degeneriert} nennen.\\
 	Für die Implementierung in Java haben wir uns für eine Library entschieden, die uns das Voronoi Diagramm berechnet. Den planaren Dualgraph berechnen wir aufbauend auf der Library, wie mn in Abb. \ref{fig3} sehen kann. Des weiteren sieht man, dass wir im Fall von degenerierten $2$-Flächen eine \texttt{EdgeFaceAction} benutzten, um die längste Kante aus dem Kreis sowie die $2$-Zelle direkt im selben Zeitschritt einfügen.
 	
 	\section{Berechnung der persistenten Homologie}
 	
 	


\end{document}