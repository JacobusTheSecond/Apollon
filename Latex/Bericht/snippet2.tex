\lstset{language=Java}
\begin{figure}
    \begin{lstlisting}[frame=single]
        public void replace(int edge, int edges) {
            if (edges.length == 0) {
                remove(edges);
                return;
            }
            boolean modified = false;
            // Replace positive and negative orientations of the edge
            int inverse = Graph.inverse(edge);
            List<Integer> positive = Util.toList(edges);
            List<Integer> negative = Util.toList(Graph.inverse(edges));
            for (int i = 0; i < size(); i++) {
                int replace = get(i);
                // Replace all occurrences with either positive or negative replacement
                if (replace == edge || replace == inverse) {
                    this.edges.remove(i);
                    this.edges.addAll(i, replace == edge ? positive : negative);
                    modified = true;
                    i += edges.length - 1;
                }
            }
            if (modified) {
                revalidate();
            }
        }

        public void remove(int... edges) {
            // Collect all edges in positive and negative orientation
            Set<Integer> set = Util.toSet(edges);
            IntStream.of(edges).map(Graph::inverse).forEach(set::add);
            if (this.edges.removeAll(set)) {
                revalidate();
            }
        }

        public void revalidate() {
            // Iterate through the circle and reduce inverse oriented consecutive edges
            for (int i = 0; i < size(); i++) {
                int j = (i + 1) % size();
                if (get(i) == Graph.inverse(get(j))) {
                    edges.remove(Math.max(i, j));
                    edges.remove(Math.min(i, j));
                    i = Math.max(0, i - (j < i ? 3 : 2));
                }
            }
        }
    \end{lstlisting}
    \caption{\textit{Codesnippet of replacement and removal of edges}}
    \label{fig5}
\end{figure}